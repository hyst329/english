\documentclass{beamer}

\usepackage{default}
\usepackage{tikz}
\usepackage{helvet}

\begin{document}
\begin{frame}
\title{Development of Machine-Readable Subsets of Natural Languages}
\author{Anna Repina \and Dmitry Galanine \and \\[8mm]Advisor: Prof. Lilia R. Sakaeva }
\institute{Institute of Computer Science, Kazan Federal University}
\date{2018--11--15}
\maketitle
\end{frame}

\begin{frame}{Goals}
\begin{itemize}
\item analysing the theory of controlled languages
\item examining existing implementations
\item implementing a custom language subset parser
\end{itemize}
\end{frame}

\begin{frame}{Basic definitions}
\textit{Controlled natural languages (CNLs)} are subsets of natural languages that are obtained by restricting the grammar and
vocabulary in order to reduce or eliminate ambiguity and complexity. 

Their purpose is usually
\begin{itemize}
\item either to improve readability for non-native speakers
\item or to enable automatic semantic analysis of the language.
\end{itemize}
\end{frame}

\begin{frame}
The automatically analysable language subsets usually have a formal logical basis, i.e. formally defined syntax and semantics.
They are often mapped to an existing formal language.
\end{frame}

\begin{frame}{Existing implementations}
An example of a controlled language subset is \textit{Attempto Controlled English (ACE)}, developed at the University of Zurich.
ACE appears perfectly natural --- it can be read and understood by any speaker of English --- it is in fact a formal language
mapped to the first-order logic.
\end{frame}

\begin{frame}
Here are some simple ACE examples
\begin{itemize}
\item \textit{Every woman is a human.}
\item \textit{A woman is a human.}
\item \textit{A man tries-on a new tie. If the tie pleases his wife then the man buys it.}
\end{itemize}
\end{frame}

\begin{frame}{Custom implementation}
\begin{tikzpicture}
\node[draw, text width = 15mm] (IF) at (0, 0) { Input file \tiny Text };
\node[draw, text width = 15mm] (P) at (4, 0) { Parser \tiny Scala };
\node[draw, text width = 20mm] (OF) at (8, 0) { Output file \tiny JSON };
\node[draw, text width = 20mm] (KB) at (4, 4) { Knowledge base \tiny SQLite };

\path [->] (IF) edge node[left] {} (P); 
\path [->] (P) edge node[left] {} (OF); 
\path [<->] (P) edge node[left] {} (KB); 
\end{tikzpicture}
\end{frame}

\begin{frame}
\begin{itemize}
\item An \textit{input file} contains some sentences in the controlled subset.
\item An \textit{output file} would contain a formal representation of the given sentences in a specified format. 
\item A \textit{parsing engine} translate sentences into syntactic constructs and prints them to an output file.
\item A \textit{knowledge base} contains information about the lexicon of the language (e.g. parts of speech).
\end{itemize}
\end{frame}

\begin{frame}
\begin{center}
\huge Thank you for your attention!
\end{center}

\end{frame}

\end{document}
