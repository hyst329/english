\documentclass{article}

\usepackage{tempora}
\usepackage[14pt]{extsizes}

\begin{document}
(Slide 1)\\

The topic of our presentation is \textit{Development of Machine-Readable Subsets of Natural Languages}.\\

(Slide 2)\\

The main goals of our work on the subject are analysis of the theoretical part (for example, formal language theory),
studying of the existing controlled languages structure and implementation, and writing an own controlled language in the
form of both specification and implementation.\\

(Slide 3)\\

So, let's begin from what the controlled language is. A \textit{controlled language} is a subset of some natural language
(for instance, English or Russian), which has formally defined grammar and syntax rules making the language unambiguously
understood (not only by humans, of course).\\

(Slide 4)\\

These subsets have many purposes, but one of our interest is enabling of automatic semantic analysis, which let the language be
machine-readable and (which is more important) machine-understandable. Such languages are usually mapped to some formal language
--- for example, first-order logic or even some programming language, which in theory will enable the non-programmers to write
instructions for the machine using natural language (or its subset at least).\\

(Slide 5)\\

An example of controlled language is the Attempto Controlled English, which is a subset of English language developed by Swiss
computer scientists and linguists.\\

(Slide 6)\\

Here you can see example sentences written in Attempto Controlled English. As you may notice, they look very natural, not
artificially constructed.\\

(Slide 7)\\

Now let's look at the architecture of a controlled language implementation.
It accepts an input file containing a list of sentences in the our controlled language and sends it to the parsing engine,
which parses the sentences, constructs the syntax trees from them and binds them contextually. The parsing engine
also interacts with the knowledge base which is used to deduce the meanings and syntactic roles of the words (except basic
auxiliary words such as \textit{if}, \textit{has}, \textit{be} etc.)\\

(Slide 8)\\

The parsing engine in written in Scala because it allows our to write very concise programming constructions which can be 
easily understood.
The parsing expression grammar 

(Slide 9)\\

Thank you for your attention. Now, any questions, please.

\end{document}